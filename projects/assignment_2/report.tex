\documentclass[12pt]{article}
\usepackage{graphicx}
\usepackage{amsmath}
\usepackage{hyperref}
\usepackage{caption}
\usepackage{geometry}
\usepackage{float}
\geometry{margin=0.5in}

\title{Optimization and Learning for Robot Control: Assignment 2}
\author{Luca Sartore - 256154}
\date{}


\begin{document}
\maketitle

\tableofcontents
% ############################ part 1 ##############################
\newpage
\section{Path Tracking}

In this first experiment we see how the simplest form of path tracking perform, as a baseline.

\subsection{Numerical results}
\begin{figure}[H]
\centering
\begin{tabular}{cc}
  \includegraphics[width=80mm]{images/base/s_vs_time.png} &   \includegraphics[width=80mm]{images/base/w_vs_time.png} \\
(a) The control variable ``s'' & (b) The control variable ``w'' \\[6pt]
 \includegraphics[width=80mm]{images/base/joint_velocity.png} &   \includegraphics[width=80mm]{images/base/joint_position.png} \\
(c) The joint velocity & (d) the joint position \\[6pt]
 \includegraphics[width=80mm]{images/base/ee_path.png} &   \includegraphics[width=80mm]{images/base/ee_pos_vs_time.png} \\
(e) Tracking on floor plane & (f) Tracking on X Y Z \\[6pt]
\multicolumn{2}{c}{\includegraphics[width=65mm]{images/base/joint_torque.png} }\\
\multicolumn{2}{c}{(g) The joint torques}
\end{tabular}
\caption{experiment result of ``Path Tracking''}
\end{figure}


\subsection{Observation on results}

We can see in sub-figures \textbf{e} and \textbf{f} that the tracing is perfect (as we would expect from path tracking)

from the sub-figures \textbf{a} and \textbf{b} we can instead note that the optimizer has decided to go slower (aka having a lower
\textbf{w} value) at the beginning, and at the end of the movement. that seem to be the most critical zones to guaranteed good tracking

% ############################ part 2 ##############################
\newpage

\section{Path Tracking + Cyclical Constrain}

In this experiment we tried to make the robot to perform a cyclical task, by adding a penalty that minimize the 
difference between the initial state and the final state of the robot.


\subsection{Numerical results}
\begin{figure}[H]
\centering
\begin{tabular}{cc}
  \includegraphics[width=80mm]{images/cyclical/s_vs_time.png} &   \includegraphics[width=80mm]{images/cyclical/w_vs_time.png} \\
(a) The control variable ``s'' & (b) The control variable ``w'' \\[6pt]
 \includegraphics[width=80mm]{images/cyclical/joint_velocity.png} &   \includegraphics[width=80mm]{images/cyclical/joint_position.png} \\
(c) The joint velocity & (d) the joint position \\[6pt]
 \includegraphics[width=80mm]{images/cyclical/ee_path.png} &   \includegraphics[width=80mm]{images/cyclical/ee_pos_vs_time.png} \\
(e) Tracking on floor plane & (f) Tracking on X Y Z \\[6pt]
\multicolumn{2}{c}{\includegraphics[width=65mm]{images/cyclical/joint_torque.png} }\\
\multicolumn{2}{c}{(g) The joint torques}
\end{tabular}
\caption{experiment result of ``Path Tracking + Cyclical Constrain''}
\end{figure}

\subsection{Observation on results}

Here the results are mixed. The joit velocity are decend, infact we can see how
they approach zero by the end of the simulation, however the 
joint position are not much different from the one of the previous experiment,
indicating that the solver is not really able to end in the same position he started. and
also keeping a perfect tracking. (that is required given in path tracking we use equality
constraints)

% ############################ part 3 ##############################
\newpage
\section{Trajectory Tracking + Cyclical Constrain}

In this section we applyed trajectory tracking instead of path tracking to observe the difference
with the previous experiment. We still maintain the so called ``cyclical constrain''


\subsection{Numerical results}
\begin{figure}[H]
\centering
\begin{tabular}{cc}
  \includegraphics[width=80mm]{images/trajectory-tracking/s_vs_time.png} &   \includegraphics[width=80mm]{images/trajectory-tracking/w_vs_time.png} \\
(a) The control variable ``s'' & (b) The control variable ``w'' \\[6pt]
 \includegraphics[width=80mm]{images/trajectory-tracking/joint_velocity.png} &   \includegraphics[width=80mm]{images/trajectory-tracking/joint_position.png} \\
(c) The joint velocity & (d) the joint position \\[6pt]
 \includegraphics[width=80mm]{images/trajectory-tracking/ee_path.png} &   \includegraphics[width=80mm]{images/trajectory-tracking/ee_pos_vs_time.png} \\
(e) Tracking on floor plane & (f) Tracking on X Y Z \\[6pt]
\multicolumn{2}{c}{\includegraphics[width=65mm]{images/trajectory-tracking/joint_torque.png} }\\
\multicolumn{2}{c}{(g) The joint torques}
\end{tabular}
\caption{experiment result of ``Trajectory Tracking + Cyclical Constrain''}
\end{figure}

\subsection{Observation on results}

Here we can see in sub-figure \textbf{e} how the tracing looses a bit of
accuracy, however we can also see in the subfigures \textbf{c} and \textbf{d}
how both joint velocity and joint position revert back to the original state
as the simulation approaches the end. This is because by choosing trajectory
tracking we are allowing the robot to do smaller deviation form his original path,
and this make it possible to optimize other tasks (such as the ``cyclical'' one).

To achieve this we had to change the ``equality'' constraints on the path, and make
that a ``cost''.
if we had left them as an equality, while removing the flexibility of 
selecting \textbf{w} the optimizer would almost certaintly fail (unless the tracking 
task was trivial).

% ############################ part 4 ##############################
\newpage

\section{Path Tracking + Cyclical Constrain + Min Time Formulation}

In this experiment we introduced a min-time Formulation constrain to the
previous path tracking task (by letting the solver optimize the time step)

\subsection{Numerical results}
\begin{figure}[H]
\centering
\begin{tabular}{cc}
  \includegraphics[width=80mm]{images/cyclical-min-time/s_vs_time.png} &   \includegraphics[width=80mm]{images/cyclical-min-time/w_vs_time.png} \\
(a) The control variable ``s'' & (b) The control variable ``w'' \\[6pt]
 \includegraphics[width=80mm]{images/cyclical-min-time/joint_velocity.png} &   \includegraphics[width=80mm]{images/cyclical-min-time/joint_position.png} \\
(c) The joint velocity & (d) the joint position \\[6pt]
 \includegraphics[width=80mm]{images/cyclical-min-time/ee_path.png} &   \includegraphics[width=80mm]{images/cyclical-min-time/ee_pos_vs_time.png} \\
(e) Tracking on floor plane & (f) Tracking on X Y Z \\[6pt]
\multicolumn{2}{c}{\includegraphics[width=65mm]{images/cyclical-min-time/joint_torque.png} }\\
\multicolumn{2}{c}{(g) The joint torques}
\end{tabular}
\caption{experiment result of ``Path Tracking + Cyclical Constrain + Min Time Formulation''}
\end{figure}

\subsection{Observation on results}

We can see how the results beryl change from the previous path-tracking Formulation.
and the reason is easily understood by printing the time-step value after the optimization process.

We noted that the solver was actually tacking more time than the previous Formulation
(using a time step of 24ms instead of 20).

The simple solution to this issue was to increase the weight of the running cost (see next section)



% ############################ part 5 ##############################
\newpage

\section{Path Tracking + Cyclical Constrain + Min Time Formulation + Tuned Weights}

This experiment has the same setup of the previous section, but the running cost's weight
was set to 1 instead of 0.1.


\subsection{Numerical results}
\begin{figure}[H]
\centering
\begin{tabular}{cc}
  \includegraphics[width=80mm]{images/cyclical-min-time-2/s_vs_time.png} &   \includegraphics[width=80mm]{images/cyclical-min-time-2/w_vs_time.png} \\
(a) The control variable ``s'' & (b) The control variable ``w'' \\[6pt]
 \includegraphics[width=80mm]{images/cyclical-min-time-2/joint_velocity.png} &   \includegraphics[width=80mm]{images/cyclical-min-time-2/joint_position.png} \\
(c) The joint velocity & (d) the joint position \\[6pt]
 \includegraphics[width=80mm]{images/cyclical-min-time-2/ee_path.png} &   \includegraphics[width=80mm]{images/cyclical-min-time-2/ee_pos_vs_time.png} \\
(e) Tracking on floor plane & (f) Tracking on X Y Z \\[6pt]
\multicolumn{2}{c}{\includegraphics[width=65mm]{images/cyclical-min-time-2/joint_torque.png} }\\
\multicolumn{2}{c}{(g) The joint torques}
\end{tabular}
\caption{experiment result of ``Path Tracking + Cyclical Constrain + Min Time Formulation + Tuned Weights''}
\end{figure}

\subsection{Observation on results}

Here the solver managed to complete the trajectory in a much quicker time,
as the time-step was optimize to be 14ms.
Here the charts look similar to the previous ones, wit the only notable difference 
being that the torques, accelerations and velocities are all slily higher,
indicating that the controller has decided get an higher penalty
in the velocities, accelerations, and controls input in order to 
minimize the penalty on the time-step



% ############################ part 6 ##############################
\newpage
\section{Trajectory Tracking + Cyclical Constrain + Min Time Formulation}

In this experiment apply the same min-time Formulation that we made in the previous
section to trajectory Tracking.

\subsection{Numerical results}
\begin{figure}[H]
\centering
\begin{tabular}{cc}
  \includegraphics[width=80mm]{images/trajectory-tracking-min-time/s_vs_time.png} &   \includegraphics[width=80mm]{images/trajectory-tracking-min-time/w_vs_time.png} \\
(a) The control variable ``s'' & (b) The control variable ``w'' \\[6pt]
 \includegraphics[width=80mm]{images/trajectory-tracking-min-time/joint_velocity.png} &   \includegraphics[width=80mm]{images/trajectory-tracking-min-time/joint_position.png} \\
(c) The joint velocity & (d) the joint position \\[6pt]
 \includegraphics[width=80mm]{images/trajectory-tracking-min-time/ee_path.png} &   \includegraphics[width=80mm]{images/trajectory-tracking-min-time/ee_pos_vs_time.png} \\
(e) Tracking on floor plane & (f) Tracking on X Y Z \\[6pt]
\multicolumn{2}{c}{\includegraphics[width=65mm]{images/trajectory-tracking-min-time/joint_torque.png} }\\
\multicolumn{2}{c}{(g) The joint torques}
\end{tabular}
\caption{experiment result of ``Trajectory Tracking + Cyclical Constrain + Min Time Formulation''}
\end{figure}

\subsection{Observation on results}

This are by far the best result we got.
We can see how the tracking is almost perfect, even tho we are using trajectory tracking instead of path tracking.
The total time required to make the maneuver is also quite low (with a time-step that was around 6ms).

However we can see from sub-plot \textbf{c} and \textbf{d} that the robot does not 
return to the original state quite as well as our previous best result did.

Joint velocity and accelerations are higher than previous tests, but not extremely higher.
This mey sound absurd at first, however we can make sense of it by locking at the joint position
sub-plot \textbf{d}.
We see how this chart look a bit different from the rest, and this imply that the robot is using a different
posture, to achieve the task.
And we can assume that this posture is better for a quick execution.

Why the solver did not use this posture for previous optimization tasks is hard to say, as
solvers like the one we used often function as a ``black box'' however we can hypnotize
that this configuration was not used in path tracking because it made it impossible to track perfectly,
and it also was not used during the previous trajectory tracking test, because the ``cyclical constrain''
had an higher weight relatively to the others, and maybe this configuration was not chosen 
because it provided a worst end position.


\end{document}