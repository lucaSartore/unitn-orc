\documentclass[12pt]{article}
\usepackage{graphicx}
\usepackage{amsmath}
\usepackage{hyperref}
\usepackage{caption}
\usepackage{geometry}
\usepackage{float}
\geometry{margin=1in}

\title{Optimization and Learning for Robot Control: Final Project}
\author{Luca Sartore - 256154}
\date{}


\begin{document}
\maketitle

\tableofcontents


\section{Introduction}

\section{Creating a dataset using Optimal Control}

\subsection{Solving the optimal control problem}

\subsection{Dataset creation}

\section{Training the critic}

\subsection{Simple Critic}

\begin{figure}[H]
\centering
\includegraphics[width=0.6\textwidth]{./images/simple_system___critic_function.png}
\caption{Baseline controller performance.}
\label{fig:simple_critic_fn}
\end{figure}


\subsection{Inertia Critic}

\begin{figure}[H]
\centering
\includegraphics[width=0.6\textwidth]{./images/inertia_system___critic_function.png}
\caption{Baseline controller performance.}
\label{fig:inertia_critic_fn}
\end{figure}


\section{Training the actor}

\subsection{Simple Actor}

\begin{figure}[H]
\centering
\includegraphics[width=0.6\textwidth]{./images/simple_system___actor_function.png}
\caption{Baseline controller performance.}
\label{fig:simpl.e_actor_fn}
\end{figure}


\subsection{Inertia Actor}


\begin{figure}[H]
\centering
\includegraphics[width=0.6\textwidth]{./images/inertia_system___actor_function.png}
\caption{Baseline controller performance.}
\label{fig:inertia_actor_fn}
\end{figure}

\section{Optimal control vs Actor Approximation}



\section{Solving time of different methods}


\end{document}